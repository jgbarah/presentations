% main.tex
% Fichero principal de transparencias (incluye a todos los demás).

% Compilar a .pdf con LaTeX (pdflatex)
% Es necesario instalar Beamer (paquete latex-beamer en Debian)
%

% Gráficos:
% Los gráficos pueden suministrarse en PNG, JPG, TIF, PDF, MPS
% Los EPS deben convertirse a PDF (usar epstopdf)
%
%\documentclass[17pt,aspectratio=169,hyperref={pdfusetitle,colorlinks,citecolor=blue,linkcolor=blue,urlcolor=blue}]{beamer}
\documentclass[17pt,aspectratio=169,hyperref={pdfusetitle,colorlinks,allcolors=olive}]{beamer}
\usetheme[orchid]{Hannover}
\beamertemplatenavigationsymbolsempty
\setbeamertemplate{headline}{}
\useoutertheme{infolines}

\usepackage{lmodern}
\usepackage[spanish]{babel}
\usepackage[utf8]{inputenc}
\usepackage{graphics}
\usepackage{multicol}
%\usepackage{amssymb} % Simbolos matematicos
%\usepackage[pdfusetitle]{hyperref}

%\usepackage{chronosys}

%% two slides per page
%\usepackage{pgfpages}
%\pgfpagesuselayout{2 on 1}[a4paper,border shrink=5mm]
%\usepackage{tikz}

\newcommand\YUGE{\fontsize{48}{60}\selectfont}

\newcommand{\secimage}{figs/bookpages}
\AtBeginSection[]
{
  {
    \usebackgroundtemplate{\includegraphics[width=\paperwidth,height=\paperheight]{\secimage}}
    \begin{frame}<beamer>

      \begin{center}
        {\YUGE\bf\insertsection}
      \end{center}
    \end{frame}
  }
  \renewcommand{\secimage}{figs/bookpages}
}


\title[softDev]{softDev: Research group on software development}
%\subtitle{}
\author[Jesus M. Gonzalez-Barahona]{Jesus M. Gonzalez-Barahona}
\institute[URJC]{Universidad Rey Juan Carlos \\
  \url{https://floss.social/@jgbarah} ~~~~~ \url{https://jgbarah.github.io/presentations}}

\date[Open Science Workshop, 2022]{\small  I Jornada Divulgación Investigación ETSIT \\
  ETSIT-URJC, Fuenlabrada, Spain, November 22nd 2022}
\begin{document}

%\begin{frame}[label=firstframe]
\begin{frame}
  \maketitle
\end{frame}


%%-----------------------------------------
\begin{frame}[fragile]
  \frametitle{Topics}

  {\Large \center
  ``Understanding software development (empirical approach)''
}

\end{frame}

%%-----------------------------------------
\begin{frame}[fragile]
  \frametitle{Topics}

  \begin{itemize}
  \item Mining software repositories
  \item Robotics software
  \item Real-time multimedia web-based applications
  \item Testing in the large
  \item Computational thinking
  \item Evaluation of software comprehension
  \item Data visualization in extended reality
  \end{itemize}
\end{frame}

%%-----------------------------------------
\begin{frame}[fragile]
  \frametitle{People: research members}

  {\footnotesize
  \begin{columns}[T]
    \begin{column}{.48\textwidth}

      \begin{itemize}
      \item González Gómez, Juan
      \item Cañas Plaza, Jose María
      \item González Barahona, Jesús
      \item Leal Algara, Elena Katia
      \item Robles Martínez, Gregorio
      \item Martín Rico, Francisco
      \end{itemize}
    \end{column}%
    \hfill%
    \begin{column}{.48\textwidth}
      \begin{itemize}
      \item Montalvo Herranz, María del Soto
      \item Gallego Carrillo, Micael
      \item Gortázar Bellas, Francisco
      \end{itemize}
    \end{column}%
  \end{columns}
}
\end{frame}


\frame{
~
\vspace{1cm}

\begin{flushright}

\includegraphics[width=2.2cm]{figs/by-sa}
 \\

\begin{footnotesize}
\copyright 2022 Jesus M. Gonzalez-Barahona. \\

\vspace{.4cm}

Some rights reserved. This document is distributed under the terms of the Creative Commons License ``Attribution-ShareAlike 4.0'',
available in \\
{\scriptsize \url{http://creativecommons.org/licenses/by-sa/4.0/}} \\

\vspace{.4cm}

This document (including source) is available from
\url{https://jgbarah.github.io/presentations}

\end{footnotesize}
\end{flushright}

}
%%

%\againframe{firstframe}

\end{document}

%%% Local Variables:
%%% mode: latex
%%% TeX-master: t
%%% End:
